%%%%%%%%%%%%%%%%%%%%%%%%%%%%%%%%%%%%%%%%%
% "ModernCV" CV and Cover Letter
% LaTeX Template
% Version 1.3 (29/10/16)
%
% This template has been downloaded from:
% http://www.LaTeXTemplates.com
%
% Original author:
% Xavier Danaux (xdanaux@gmail.com) with modifications by:
% Vel (vel@latextemplates.com)
%
% License:
% CC BY-NC-SA 3.0 (http://creativecommons.org/licenses/by-nc-sa/3.0/)
%
% Important note:
% This template requires the moderncv.cls and .sty files to be in the same 
% directory as this .tex file. These files provide the resume style and themes 
% used for structuring the document.
%
%%%%%%%%%%%%%%%%%%%%%%%%%%%%%%%%%%%%%%%%%

%----------------------------------------------------------------------------------------
%	PACKAGES AND OTHER DOCUMENT CONFIGURATIONS
%----------------------------------------------------------------------------------------

\documentclass[11pt,a4paper,sans]{moderncv} % Font sizes: 10, 11, or 12; paper sizes: a4paper, letterpaper, a5paper, legalpaper, executivepaper or landscape; font families: sans or roman

\moderncvstyle{classic} % CV theme - options include: 'casual' (default), 'classic', 'oldstyle' and 'banking'
\moderncvcolor{blue} % CV color - options include: 'blue' (default), 'orange', 'green', 'red', 'purple', 'grey' and 'black'
\usepackage{multicol}
\usepackage{lipsum} % Used for inserting dummy 'Lorem ipsum' text into the template
\usepackage[hyperref]{}
\usepackage[scale=0.9]{geometry} % Reduce document margins
%\setlength{\hintscolumnwidth}{3cm} % Uncomment to change the width of the dates column
%\setlength{\makecvtitlenamewidth}{10cm} % For the 'classic' style, uncomment to adjust the width of the space allocated to your name

%----------------------------------------------------------------------------------------
%	NAME AND CONTACT INFORMATION SECTION
%----------------------------------------------------------------------------------------

\firstname{He} % Your first name
\familyname{Bai (Richard)} % Your last name

% All information in this block is optional, comment out any lines you don't need
%\title{Curriculum Vitae}
\address{200 Ring Rd, Waterloo}{ON, Canada N2L 3G1}
\mobile{(+1) 548 333 6369 }
%\phone{(000) 111 1112}
%\fax{(000) 111 1113}
\email{he.bai@uwaterloo.ca}
%\email{asinha@mt.iitr.ac.in}
\homepage{uwaterloo.ca/scholar/h32bai}{https://uwaterloo.ca/scholar/h32bai} % The first argument is the url for the clickable link, the second argument is the url displayed in the template - this allows special characters to be displayed such as the tilde in this example
%\extrainfo{\href{https://github.com/richardbaihe}{Github: richardbaihe}}


%\photo[70pt][0.4pt]{pictures/picture} % The first bracket is the picture height, the second is the thickness of the frame around the picture (0pt for no frame)
%\quote{"A witty and playful quotation" - John Smith}

%----------------------------------------------------------------------------------------

\begin{document}

%----------------------------------------------------------------------------------------
%	COVER LETTER
%----------------------------------------------------------------------------------------

% To remove the cover letter, comment out this entire block

%\recipient{HR Department}{Corporation\\123 Pleasant Lane\\12345 City, State} % Letter recipient
%\date{\today} % Letter date
%\opening{Dear Sir or Madam,} % Opening greeting
%\closing{Sincerely yours,} % Closing phrase
%\enclosure[Attached]{curriculum vit\ae{}} % List of enclosed documents
%
%\makelettertitle % Print letter title
%
%\lipsum[1-2] % Dummy text
%\lipsum[4] % Dummy text
%
%\makeletterclosing % Print letter signature
%
%\newpage

%----------------------------------------------------------------------------------------
%	CURRICULUM VITAE
%----------------------------------------------------------------------------------------

\makecvtitle % Print the CV title

%----------------------------------------------------------------------------------------
%	EDUCATION SECTION
%----------------------------------------------------------------------------------------

\section{Education}
\cventry{Start from 09/2019}{University of Waterloo}{\newline Computer Science}{PhD}{\newline Supervisor: Ming Li}{}

\cventry{09/2017--07/2019}{Institute of Automation, Chinese Academy of Sciences}{\newline Pattern Recognition and Intelligent System}{Master of Science}{\newline Supervisor: Yu Zhou, Chengqing Zong}{} 

% \cventry{09/2013--07/2017}{Harbin Institute of Technology}{\newline Control Science and Engineering}{Bachelor of Engineering}{\newline GPA: 4.07 /4.30. Rank: top 10\%}{}

% Arguments not required can be left empty
%\cventry{2016}{D.A.V. Public School, Bistupur}{\newline Mathematics and Computer Science}{Class 12}{}{GPA -- 94.00\%}  % 
%\cventry{2014}{A.D.L.S. Sunshine School, Sakchi}{\newline Mathematics and Computer Science}{Class 10}{}{GPA -- 95.20\%}  %

%----------------------------------------------------------------------------------------
%	WORK EXPERIENCE SECTION
%----------------------------------------------------------------------------------------

\section{Research Experience}


\cventry{May 2022-- Aug. 2022}{ML Research Intern}{\textsc{Apple, Cupertino}}{}{}{
\begin{itemize}
\item My research project would be the large language model pre-training for text generation. We will answer this research question: can multiple experts outperform a super expert in the context of GPT3 pre-training?
\end{itemize}}

\cventry{Sep. 2021-- Mar. 2022}{NLP Research Intern}{\textsc{Baidu USA, Sunnyvale}}{}{}{
\begin{itemize}
\item Speech-text pre-training for speech editing and prompt-based one-shot speech synthesis.
\end{itemize}}

\cventry{May 2021-- Jul. 2021}{Research Intern}{\textsc{Microsoft Research, Montreal}}{}{}{
\begin{itemize}
\item Language model trained with curriculum learning and hypernym prediction task.
\end{itemize}}

\cventry{Jul. 2018-- Oct. 2018}{NLP Research Intern}{\textsc{IBM Research, Beijing}}{}{}{
\begin{itemize}
\item Chinese Openai-GPT model pre-training.
\end{itemize}}

%------------------------------------------------


\section{Pre-printed papers}
\cvitem{$\bullet$}{\textbf{He Bai}, Renjie Zheng, Xintong Li, Junkun Chen, Liang Huang. \textbf{Fused Acoustic and Text Pretraining for Speech Synthesis and Editing.} Submitted to  \textit{\textbf{ICML} 2022, under review~(full paper).}}


\section{Publications}

\cvitem{$\bullet$}{\textbf{He Bai}, Tong Wang, Alessandro Sordoni, Peng Shi. \textbf{Better Language Model with Hypernym Class Prediction.} \textit{\textbf{ACL} 2022~(full paper).}}

\cvitem{$\bullet$}{Peng Shi, Rui Zhang, \textbf{He Bai}, Jimmy Lin. \textbf{Cross-Lingual Training with Dense Retrieval for Document Retrieval.} \textit{\textbf{EMNLP-MSR} 2021~(workshop paper).}}


\cvitem{$\bullet$}{\textbf{He Bai}, Peng Shi, Jimmy Lin, Luchen Tan, Kun Xiong, Wen Gao, Jie Liu, Ming Li. \textbf{Semantics of the Unwritten: The Effect of End of Paragraph and Sequence Tokens on Text Generation.} \textit{\textbf{ACL-SRW} 2021~(workshop paper).}}


\cvitem{$\bullet$}{\textbf{He Bai}, Peng Shi, Jimmy Lin, Yuqing Xie, Luchen Tan, Kun Xiong, Wen Gao, Ming Li.
 \textbf{Segatron: Segment-aware Transformer for Language Modeling and Understanding.} \textit{\textbf{AAAI} 2021~(full paper).}}
 
\cvitem{$\bullet$}{Peng Shi, \textbf{He Bai}, Jimmy Lin.\textbf{Cross-Lingual Training of Neural Models for Document Ranking.} \textit{Findings of \textbf{EMNLP} 2020~(short paper).}} 


% \cvitem{$\bullet$}{Minghan Li, \textbf{He Bai}, Luchen Tan, Kun Xiong, Jimmy Lin \textbf{Latte-Mix: Measuring Sentence Semantic Similarity with Latent Categorical Mixtures.} \textit{arXiv preprint arXiv:2010.11351.}}

\cvitem{$\bullet$}{\textbf{He Bai}, Yu Zhou, Jiajun Zhang and Chengqing Zong. \textbf{Memory Consolidation for Contextual Spoken Language Understanding with Dialogue Logistic Inference.} \textit{\textbf{ACL} 2019~(short paper).}}

\cvitem{$\bullet$}{\textbf{He Bai}, Yu Zhou, Jiajun Zhang, Liang Zhao, Mei-Yuh Hwang and Chengqing Zong. \textbf{Source Critical Reinforcement Learning for Transferring Spoken Language Understanding to a New Language.} \textit{\textbf{COLING} 2018~(full paper).}}

% \section{Research}
% \cventry{}{Alignment-Aware Acoustic and Text Modeling}{Liang Huang}{Baidu USA}{}{
% \begin{itemize}
% 	\item We propose an alignment-aware acoustic-text pre-training method for speech editing, and outperforms the current best system. \href{https://educated-toothpaste-462.notion.site/ICML-Demo-fdacf73d17904fad8901548504aece9d}{\color{blue}{\underline{demo link}}}
% 	\item We further find our pre-trained model can be used for prompt-based speech synthesis, for example, unseen speaker TTS and speech cloning. \href{https://educated-toothpaste-462.notion.site/One-shot-Speech-Synthesis-with-Acoustic-Prompt-8a468383d5384591975d21f98feff736}{\color{blue}{\underline{demo link}}}
% \end{itemize}
% }
% \cventry{}{Hypernym-Instructed Language Modeling}{Tong Wang \& Alessandro Sordoni}{Microsoft Research}{}{
% \begin{itemize}
% 	\item We propose to train Transformer-based language model with WordNet's hypernym relationships through curriculum learning.
% 	\item Our method consistently reduces perplexity over various large, highly-performant LMs on WikiText-103 and ARXIV datasets, especially the perplexity of low-frequency nouns.
% \end{itemize}
% }
% \cventry{}{Segment-aware Language Modeling}{Ming Li}{University of Waterloo}{}{
% \begin{itemize}
% 	\item We propose a segment-aware transformer, Segatron, for language modeling, which achieves SOTA results on the WikiText-103 dataset. 
% 	\item BERT pre-trained with Segatron outperformers the original BERT on GLUE, SQUAD, and RACE. Our model also outperforms Sentence-RoBERTa-large on sentence representation learning task~(STS).
% \end{itemize}
% }

% \cventry{}{Multi-turn Spoken Language Understanding}{Chengqing Zong}{Chinese Academy of Sciences}{}{
% \begin{itemize}
% 	\item We propose an auxiliary training task for multi-turn spoken language understanding: conversation history sequence ordering. The SLU model trained with our proposed multi-task training gets significant improvement on KVRET dataset.
% \end{itemize}
% }

% \cventry{}{Cross-lingual Spoken Language Understanding}{Mei-yuh Hwang}{University of Washington}{}{
% \begin{itemize}
% 	\item In this project, we transfer spoken language understanding dataset from Chinese to English with RL-based machine translation method, which outperforms both alignment-based and other translation-based methods.
% \end{itemize}
% }



\section{Program Committees and Peer Review}
\cventry{2021-now}{Association for 
Computational Linguistics Rolling Review~(ARR)}{}{}{}{}
\cventry{2022}{International Conference on Computational Linguistic~(COLING)}{}{}{}{}
\cventry{2022}{Workshop on Multilingual Information Access~(MIA)}{}{}{}{}
\cventry{2021}{Annual Meeting of the Association for Computational Linguistics~(ACL)}{}{}{}{}
\cventry{2020}{Annual Meeting of the Association for Computational Linguistics~(ACL)}{}{}{}{}
\cventry{2020}{AAAI Conference on Artificial Interlligence~(AAAI)}{}{}{}{}
\cventry{2020}{International Conference on Computational Linguistics~(COLING)}{}{}{}{}
\cventry{2019}{Conference on Empirical Methods in Natural Language Processing~(EMNLP)}{}{}{}{}
% \cvitem{Reviewer}{ARR, ACL 2021, ACL 2020, AAAI 2020, COLING 2020, EMNLP 2019}
% \cvitem{Workshop PC} {MIA 2022}

\section{Insterests}
\cvitem{Research}{Language Sequence Modeling, Unsupervised Learning, Multilingual NLP}
\cvitem{Personal}{Traveling, Cooking}

%----------------------------------------------------------------------------------------
%	INTERESTS SECTION
%----------------------------------------------------------------------------------------

%\section{Interests}

%\renewcommand{\listitemsymbol}{-~} % Changes the symbol used for lists

%\cvlistdoubleitem{Cycling}{Hiking}
%\cvlistdoubleitem{Sketching}{Gaming}
%\cvlistitem{Quizzing}
%----------------------------------------------------------------------------------------

%\section{References}
%
%\begin{multicols}{2}
%\cventry{}{K.S Suresh}{\newline Assistant Professor}{\newline Metallurgical and Materials Engg., IIT Roorkee}{\newline suresfmt@iitr.ac.in}{}
%\columnbreak
%\cventry{}{Anu Chandra}{\newline CEO}{\newline Ryelore AI}{\newline anu@ryelore.com}{ }
%\end{multicols}

%\cventry{}{Arpit Gupta}{\newline VP Engineering}{\newline Antriex IT Services}{\newline arpit.gupta@antmex.com}{}

%\cventry{}{B.S.S Daniel}{\newline Professor}{\newline Metallurgical and Materials Engg., IIT Roorkee}{\newline s4danfmt@iitr.ac.in}{ }
\end{document}
